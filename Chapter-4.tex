% !TEX root = main.tex
\chapter{Related Work}
\label{Chapter:RelatedWork}

There have been a large number of studies done on open source software, particularly the mining of GitHub data to gain a better understanding of how the widely popular tool is being used across the globe. Due to the incredibly large nature of GitHub as a platform (as highlighted in Chapter \ref{Chapter:Background}), it is important for researchers to understand first the questions they want to answer and second the various options they have available to collect GitHub data prior to starting the mining process. Cosentino, Luis and Cabot collected 231 works on mining GitHub, eventually narrowing these down to a set of 93 which they use to try and better understand how GitHub repositories are being mined, how the data ends up being used and how associated limitations are presented and addressed \cite{cabot_should_2016}. They discuss overall limitations with the use of various GitHub datasets and the fact that they are not always current or consistent. They also found that two thirds of the 93 selected works did not provide enough information for future comparative studies to replicate the work. They discuss limitations of their own study and conclude by highly suggesting that researchers ensure datasets and instructions to replicate studies on mining GitHub are shared with the research community. 

Prior to beginning this research on the RTP area, ideas on what we wanted to focus on were formed through the perusal of various published papers. Related work on topics such as the impact of geography on contributions in GitHub, determination of local skill sets based off GitHub data, programming language trends, the impact social networking features have on users as well as a study that looked to determine the promises and perils of mining GitHub through user surveys will be cited and discussed further in this chapter. The most important piece of research that we initially focused on was the GHTorrent project created by G. Gousios. Much of the information sourced from this project's public research is already outlined in Chapter \ref{Chapter:Background}. We were able to use the GHTorrent data source as a means to quickly get started in searching for RTP specific data on users, their repositories and overall open source activity, which we later use to answer questions on what the social engineering demographic looks like in RTP.

A study was done by a group of researchers (including Gousios from GHTorrent) on the “promises and perils” of mining GitHub \cite{kalliamvakou_promises_2014}. At that point, there were no known studies of the quality and properties of data available through GitHub. This research was focused around trying to gain a better understanding of how users take advantage of the various capabilities that they are exposed to within GitHub, particularly committing, submitting pull requests and issues. Out of the 1,000 surveys sent to active users who had listed their e-mail address on their GitHub account, only 240 responses came back. Although the sample was small, the researchers were able to collect some valuable information which they share as part of their research. This study concluded, in summary, by showing that most repositories are inactive and only created for personal reasons. We can conclude similar results as part of our research on the RTP area given the number of repositories that were created and never touched again. These researchers also found out through a survey, that many developers host their code on GitHub for the sole reason of free hosting, having no desire to ever open it up for collaboration. In addition, 38\% of users involved in the survey said they use GitHub primarily for their own projects, with no intention of collaboration with other developers. Due to these findings, the authors provide recommendations based on their research to suggest the best way to study GitHub data as a whole by explaining in detail some of the different workflow use cases that they are aware of for personal and project work. 

In today's communities (from our experience), development of software is quite often split between different regional teams. As audio/video collaboration technologies continue to advance, the distance of these teams matter less. However, there are still many instances where regional clusters contain concentrated software engineering industries---RTP being an example as we are demonstrating through this research. Takhteyev and Hilts did research on the geography of open source software through GitHub \cite{takhteyev_investigating_2010}. These researchers came up with a unique method to study regional teams and their associated involvement in open source projects. They started with a single account (one of GitHub's founders) and began collecting more accounts and information based off of their connections, ending up with a sample of 70,414 accounts. As we mentioned previously, the location field in a GitHub user's public profile is free text and hence difficult to query directly. This team created a method to code the location using Geonames.org, Yahoo's GeoAPI, as well as some manual intervention. They were able to conclude (based on sample accounts) that 39\% of all GitHub accounts are located in the United States, the remaining 61\% being spread across the globe (second largest country being the United Kingdom at 7\%). The top 5 clusters identified in the United States were San Francisco, New York, Boston, Seattle, and Chicago.

David Rusk and Yvonne Coady from the University of Victoria did some research on analyzing the popularity of programming languages in their local community (Victoria, BC, Canada) in 2014 \cite{rusk_location-based_2014}. One of the goals to this research was to give employers as well as potential employees a better idea of what the local skill sets look like through a talent pool repository as well as an overview of the common technologies used in any given location. This team created a tool which pulled data out of the GitHub REST API and dumped it into a MongoDB instance. It is unclear why they did not use the GHTorrent data source as it was cited as related work and all of this information could have been pulled directly from the GHTorrent MySQL instance itself. They were able to create some visualizations similar to what was done during the data analysis phase of this project and provided these to local developers and businesses for feedback. They discussed that most of the developers found the information quite helpful, some being a bit concerned with developer privacy given the researchers were publishing names and other bits of personal information within the project itself. Others pushed back and highlighted the fact that these profiles were already public to begin with and that listing them within the project was a required feature. Employer feedback was also positive, they were happy with the fact that they could find developers with specific skills, code samples they've written, as well as contact information all in one place. Their aim was to be able to expand this to other locations as part of future work. As our research looks at similar metrics for the RTP region, publishing this data could prove useful in the future. 

In 2013, Begel, Bosch and Storey conducted interviews with 4 leaders from the software development industry to try and gain a better understanding around the social networking aspects of open source software development \cite{begel_social_2013}. Brian Doll, an engineer in the marketing space for GitHub was interviewed as part of this initiative. We found some of his answers particularly interesting and could relate them back to some of the questions we were looking to answer as part of this research. Storey asked Doll, "How does social networking play a role in the services you provide?" Doll then started describing a recent e-mail he had received from a past colleague who had mentioned he was following his activity on GitHub, such as the repositories he had starred. In this way, users are able to keep up with various projects that their connections are watching or actively participating in. We explain how often RTP users are utilizing this feature as part of our local research. Doll was also asked how other relationships were formed on GitHub, outside of the “stargazing” technique. He started describing some of the use cases for creating a GitHub “organization” user type – which is something we look at the usage of for our local research. Doll mentions that it is quite common for projects to be managed in this way on GitHub because "it's the cleanest way for them to give permissions to several developers with different levels of access to the code." Another important point that Doll makes in this interview is the importance that GitHub sees in ensuring users can “put a face to the name of the project's main developer.” This way, the user who is doing the bulk of the work is getting the actual credit for it. For this reason, GitHub keeps the user login ID in the URL structure of each repository, and also makes heavy use of avatars in activity feeds. This point is interestingly related to the current controversy over privacy between GHTorrent project and GitHub users \cite{_issue_????}. Doll also discusses his opinion around the benefit of open source software development for people looking to get a job within a programming company, regardless of experience, and discusses articles that he has read which claim GitHub as being “the new resume for programmers”. 
