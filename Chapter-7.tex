% !TEX root = main.tex
\chapter{Conclusion}
\label{Chapter:Conclusion}

The purpose of this research was to shed light on the open source activities in the Research Triangle Park region using information mined from  GitHub. We began the data collection phase by gathering metrics around how many people in the RTP area were involved in open source projects (from both personal and organizational levels). This information was then used to explore repositories developed out of the RTP area, looking into characteristics such as the programming languages used, their associated activity, and current relevance of the various projects. Lastly, we explored how RTP users were involved in the social media aspects of GitHub. In summary, we found that, as a whole, the RTP region is not heavily involved in the open source community, but we were able to identify a number of users that were prominent on the platform both from the single user and organizational perspective.

As part of future research, it would be interesting to extend these experiments to cover other locations with a similar technology industry profile, providing additional context for the current results and providing insight into how open source development differs in different parts of the United States and in technology hubs in other countries. It would also be interesting to add a temporal aspect to this research, exploring how open source development activities change over time. Teams and individual developers could also use this work to find out which cities have the most users, repositories, open source activity, and overall popularity, potentially by providing this data through a portal. Lastly, it would also be interesting to compare findings such as the most popular programming languages to the job requirements in the local markets to see if any trends can be identified. The scripts developed for this thesis, used as part of the data collection and analysis phases, are themselves available on GitHub under an open-source license.
