% !TEX root = main.tex
\chapter{Introduction}
\label{Chapter:Introduction}

Open source software development is continuing to grow within information technology communities. The data capable of being extracted from collaborative development platforms gives researchers an opportunity to answer many questions about open source project activity and trends. However, at this point in time, we know little about how people are participating in open source development from a regional perspective. This research is aimed at studying open source activities through public data within GitHub \cite{_github_2016} to understand what the open source software engineering demographic looks like in Research Triangle Park (RTP), NC, one of the most prominent research parks in the world. 

GitHub is a widely used platform for open source software development and collaboration. Its extensive features give software developers the capability to join forces with others from all over the world on a variety of projects, from small projects to major projects that are being used at the enterprise level. GitHub provides a public REST API which grants researchers the opportunity to mine and study this data from an empirical perspective. Using the GHTorrent Project \cite{gousios_ghtorrent:_2012}, \cite{gousios_ghtorent_2013} as our foundation for information, we focus on answering a few key questions throughout this research:
\begin{enumerate}
\item What do the metrics look like around individual users and organizations and their involvement in open source projects in the RTP area? 
\item What do the overall numbers look like with regards to projects being developed by users and organizations in the RTP region? How active are these projects, and are they still relevant since inception?  
\item How do users and organizations in the RTP region use the social media features available on GitHub?
\end{enumerate}

While the answers to these questions have focused on the RTP region, the scripts developed for this empirical study can also, with some generalization, be used to study other geographic areas, which will enable future research including answering similar questions for other regions and comparing multiple regions. Chapter 2 will go through key reasons on why RTP was chosen as a strategic location for this research. Chapter 3 will provide more information around GitHub, with an overview of terms and associated workflows, as well as an overview of the GHTorrent project. Chapter 4 will discuss related research on this topic, particularly studies into open source data. Chapter 5 and 6 will go through the methodologies used for data collection and then provide analysis of the RTP data collected. Lastly, we will wrap up in Chapter 7 with closing remarks about this research and ideas for future research. 

\paragraph{Research Contribution:} Through empirical techniques applied to data mined from project repositories on GitHub, this thesis quantitatively shows how both individual developers and organizations participate in open source software development in the Research Triangle Park region of North Carolina. The process used to extract and compute over the repository data has been scripted to ensure the reported results can be easily replicated, while the scripts themselves have been designed to enable future studies of other technology regions as well as comparative studies of multiple regions. The data collection tools and techniques created to enable this research are publicly available on GitHub (User ID: LindseyKLanier, \cite{_lindsey_github}) under an open-source license. 